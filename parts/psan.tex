% Définition du Process Hitting + sortes coopératives

\begin{frame}
  \frametitle{Parametric Stochastic Automata Networks}
  %\framesubtitle{\tcite{Paulev\'e et al. 2012}}


\begin{columns}
\begin{column}{0.6\textwidth}
% 1 : Sortes
\only<1>{
\tikzstyle{process}=[circle,minimum size=15pt,font=\footnotesize,inner sep=1pt]
\tikzstyle{tick label}=[color=white, font=\footnotesize]
\tikzstyle{tick}=[transparent]
\tikzstyle{hit}=[transparent]
\tikzstyle{selfhit}=[transparent, min distance=30pt,curve to]
\tikzstyle{bounce}=[transparent]
\tikzstyle{hlhit}=[transparent]
\tikzstyle{local transitions}=[transparent]
\begin{center}\scalebox{\scaleex}{
\begin{tikzpicture}
\exandef
\end{tikzpicture}
}\end{center}
}

% 2 : Processus
\only<2>{
\tikzstyle{process}=[circle,draw,minimum size=15pt,font=\footnotesize,inner sep=1pt]
\tikzstyle{tick label}=[font=\footnotesize]
\tikzstyle{tick}=[densely dotted]
\tikzstyle{hit}=[transparent]
\tikzstyle{selfhit}=[transparent, min distance=30pt,curve to]
\tikzstyle{bounce}=[transparent]
\tikzstyle{hlhit}=[transparent]
\tikzstyle{local transitions}=[transparent]
\begin{center}\scalebox{\scaleex}{
\begin{tikzpicture}
\exandef
\end{tikzpicture}
}\end{center}
}

% 3 : États
\only<3>{
\tikzstyle{hit}=[transparent]
\tikzstyle{selfhit}=[transparent, min distance=30pt,curve to]
\tikzstyle{bounce}=[transparent]
\tikzstyle{hlhit}=[transparent]
\tikzstyle{local transitions}=[transparent]
\begin{center}\scalebox{\scaleex}{
\begin{tikzpicture}
\exandef

\TState{3}{a_0,b_0,c_0}
\end{tikzpicture}
}\end{center}
}

% 4 : Actions
\only<4->{
\tikzstyle{tick}=[densely dotted]
\tikzstyle{hit}=[->,>=angle 45]
\tikzstyle{selfhit}=[min distance=30pt,curve to]
\tikzstyle{bounce}=[densely dotted,>=stealth',->]
\tikzstyle{hlhit}=[very thick]
\begin{center}\scalebox{\scaleex}{
\begin{tikzpicture}
\exandef
\TState{4}{a_0,b_0,c_0}
\TState{5}{a_1,b_0,c_0}
\TState{6}{a_0,b_0,c_0}
\TState{7}{a_2,b_0,c_0}
\TState{8}{a_2,b_1,c_0}
\end{tikzpicture}
}\end{center}
}
\end{column}

\begin{column}{0.4\textwidth}
\begin{figure}[p]
\centering

\scalebox{0.4}{
\only<4>{
\tikzstyle{arc0}=[transparent]
\tikzstyle{nd0}=[]
\tikzstyle{arc1}=[transparent]
\tikzstyle{nd1}=[transparent]
\tikzstyle{arc2}=[transparent]
\tikzstyle{nd2}=[transparent]
\tikzstyle{arc3}=[transparent]
\tikzstyle{nd3}=[transparent]
\tikzstyle{arc4}=[transparent]
\tikzstyle{nd4}=[transparent]
\tikzstyle{arc5}=[transparent]
\tikzstyle{nd5}=[transparent]
\input{figures/sg-example3.pgf}
}
\only<5>{
\tikzstyle{arc0}=[->]
\tikzstyle{nd0}=[]
\tikzstyle{arc1}=[->]
\tikzstyle{nd1}=[]
\tikzstyle{arc2}=[transparent]
\tikzstyle{nd2}=[transparent]
\tikzstyle{arc3}=[transparent]
\tikzstyle{nd3}=[transparent]
\tikzstyle{arc4}=[transparent]
\tikzstyle{nd4}=[transparent]
\tikzstyle{arc5}=[transparent]
\tikzstyle{nd5}=[transparent]
\input{figures/sg-example3.pgf}
}
\only<6>{
\tikzstyle{arc0}=[->]
\tikzstyle{nd0}=[]
\tikzstyle{arc1}=[->]
\tikzstyle{nd1}=[]
\tikzstyle{arc2}=[->]
\tikzstyle{nd2}=[]
\tikzstyle{arc3}=[transparent]
\tikzstyle{nd3}=[transparent]
\tikzstyle{arc4}=[transparent]
\tikzstyle{nd4}=[transparent]
\tikzstyle{arc5}=[transparent]
\tikzstyle{nd5}=[transparent]
\input{figures/sg-example3.pgf}
}
\only<7>{
\tikzstyle{arc0}=[->]
\tikzstyle{nd0}=[]
\tikzstyle{arc1}=[->]
\tikzstyle{nd1}=[]
\tikzstyle{arc2}=[->]
\tikzstyle{nd2}=[]
\tikzstyle{arc3}=[->]
\tikzstyle{nd3}=[]
\tikzstyle{arc4}=[transparent]
\tikzstyle{nd4}=[transparent]
\tikzstyle{arc5}=[transparent]
\tikzstyle{nd5}=[transparent]
\input{figures/sg-example3.pgf}
}
\only<8>{
\tikzstyle{arc0}=[->]
\tikzstyle{nd0}=[]
\tikzstyle{arc1}=[->]
\tikzstyle{nd1}=[]
\tikzstyle{arc2}=[->]
\tikzstyle{nd2}=[]
\tikzstyle{arc3}=[->]
\tikzstyle{nd3}=[]
\tikzstyle{arc4}=[->]
\tikzstyle{nd4}=[]
\tikzstyle{arc5}=[transparent]
\tikzstyle{nd5}=[transparent]
\input{figures/sg-example3.pgf}
}
\only<9>{
\tikzstyle{arc0}=[->]
\tikzstyle{nd0}=[]
\tikzstyle{arc1}=[->]
\tikzstyle{nd1}=[]
\tikzstyle{arc2}=[->]
\tikzstyle{nd2}=[]
\tikzstyle{arc3}=[->]
\tikzstyle{nd3}=[]
\tikzstyle{arc4}=[->]
\tikzstyle{nd4}=[]
\tikzstyle{arc5}=[->]
\tikzstyle{nd5}=[]
\input{figures/sg-example3.pgf}
}
}

\end{figure}

\end{column}
\end{columns}
%\medskip

\begin{liste}
  \item \tval{Automata}: components \qex{$a$, $b$, $c$}
\pause[2]
  \item \tval{local states}: levels of expression \qex{$c_0$, $c_1$, $c_2$}
\pause[3]
  \item \tval{States}: sets of active local states%
  \only<3-4>{\qex{$\PHetat{a_0, b_0, c_0}$}}%
  \only<5>{\qex{$\PHetat{a_1, b_0, c_0}$}}%
  \only<6>{\qex{$\PHetat{a_0, b_0, c_0}$}}%
  \only<7>{\qex{$\PHetat{a_2, b_0, c_0}$}}%
  \only<8>{\qex{$\PHetat{a_2, b_1, c_0}$}}
\pause[4]
  \item \tval{Transitions}: dynamics \qex{\only<5>{\underline}{$t_1 = \trans{a_0}{a_1}{b_0}$}, \only<6>{\underline}{$t_2 = \trans{a_1}{a_0}{}$}, \only<7>{\underline}{$t_3 = \trans{a_0}{a_2}{b_0,c_0}$}, \only<8>{\underline}{$t_4 = \trans{b_0}{b_1}{}$}}
\end{liste}
%une seule transition à la fois
%on peut avoir un choix il faut bien expliquer ça.
\end{frame}

\begin{comment}
\begin{frame}
  \frametitle{Parametric Stochastic Automata Networks}
  %\framesubtitle{\tcite{Paulev\'e et al. 2012}}
\begin{columns}
\begin{column}{0.6\textwidth}

% Automate sans les rates
\only<1->{
\tikzstyle{tick}=[densely dotted]
\tikzstyle{hit}=[->,>=angle 45]
\tikzstyle{selfhit}=[min distance=30pt,curve to]
\tikzstyle{bounce}=[densely dotted,>=stealth',->]
\tikzstyle{hlhit}=[very thick]
\begin{center}\scalebox{\scaleex}{
\begin{tikzpicture}
\exandef
\end{tikzpicture}
}\end{center}
}
\end{column}

\begin{column}{0.4\textwidth}
\begin{figure}[p]
\centering

\scalebox{0.4}{

\only<1->{
\tikzstyle{arc0}=[->]
\tikzstyle{nd0}=[]
\tikzstyle{arc1}=[->]
\tikzstyle{nd1}=[]
\tikzstyle{arc2}=[->]
\tikzstyle{nd2}=[]
\tikzstyle{arc3}=[->]
\tikzstyle{nd3}=[]
\tikzstyle{arc4}=[->]
\tikzstyle{nd4}=[]
\tikzstyle{arc5}=[->]
\tikzstyle{nd5}=[]
\input{figures/sg-example3.pgf}
}
}

\end{figure}

\end{column}
\end{columns}

\begin{liste}
  \item \tval{Automata}: components \qex{$a$, $b$, $c$}
  \item \tval{local states}: levels of expression \qex{$c_0$, $c_1$, $c_2$}
  \item \tval{States}: sets of active local states%
  \only<1->{\qex{$\PHetat{a_2, b_1, c_0}$}}
  \item \tval{Transitions}: dynamics \qex{\only<1->{$t_1 = \trans{a_0}{a_1}{b_0}$}, \only<1->{$t_2 = \trans{a_1}{a_0}{}$}, \only<1->{$t_3 = \trans{a_0}{a_2}{b_0,c_0}$}, \only<1->{$t_4 = \trans{b_0}{b_1}{}$}}
\end{liste}
\end{frame}
\end{comment}

\begin{frame}
  \frametitle{Parametric Stochastic Automata Networks}
  %\framesubtitle{\tcite{Paulev\'e et al. 2012}}
\begin{columns}
\begin{column}{0.6\textwidth}

% Automate avec les rates
\only<1->{
\tikzstyle{tick}=[densely dotted]
\tikzstyle{hit}=[->,>=angle 45]
\tikzstyle{selfhit}=[min distance=30pt,curve to]
\tikzstyle{bounce}=[densely dotted,>=stealth',->]
\tikzstyle{hlhit}=[very thick]
\begin{center}\scalebox{\scaleex}{
\begin{tikzpicture}
\exsandef
\end{tikzpicture}
}\end{center}
}
\end{column}

\begin{column}{0.4\textwidth}
\begin{figure}[p]
\centering

\scalebox{0.4}{

\only<1->{
\tikzstyle{arc0}=[->]
\tikzstyle{nd0}=[]
\tikzstyle{arc1}=[->]
\tikzstyle{nd1}=[]
\tikzstyle{arc2}=[->]
\tikzstyle{nd2}=[]
\tikzstyle{arc3}=[->]
\tikzstyle{nd3}=[]
\tikzstyle{arc4}=[->]
\tikzstyle{nd4}=[]
\tikzstyle{arc5}=[->]
\tikzstyle{nd5}=[]
\input{figures/sg-example3-rate.pgf}
}
}

\end{figure}

\end{column}
\end{columns}

\begin{liste}
  \item \tval{Automata}: components \qex{$a$, $b$, $c$}
  \item \tval{local states}: levels of expression \qex{$c_0$, $c_1$, $c_2$}
  \item \tval{States}: sets of active local states%
  \only<1->{\qex{$\PHetat{a_2, b_1, c_0}$}}
  \item \tval{Transitions}: dynamics \qex{\only<1->{$t_1 = \trans{a_0}{a_1}{b_0,\mathbf{\color{Maroon} 2}}$}, \only<1->{$t_2 = \trans{a_1}{a_0}{\mathbf{\color{Maroon} 1}}$}, \only<1->{$t_3 = \trans{a_0}{a_2}{b_0,c_0,\mathbf{\color{Maroon} 2}}$}, \only<1->{$t_4 = \trans{b_0}{b_1}{\mathbf{\color{Maroon} 3}}$}}
\end{liste}
\end{frame}


