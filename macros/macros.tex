% Commande À FAIRE
\usepackage{color} % Couleurs du texte
%\newcommand{\afaire}[1]{\textcolor{red}{[À FAIRE : #1]}}
%\newcommand{\todo}[1]{\textcolor{red}{\textbf{[TODO\ifthenelse{\equal{#1}{}}{}{: #1}]}}}



\colorlet{couleurtheme}{gray}  % Couleur principale du thème
\colorlet{couleurcit}{darkcyan}  % Couleur des citations
\colorlet{couleurex}{blue}  % Couleur des exemples
\colorlet{couleurredex}{red}  % Couleur des exemples
\colorlet{couleurliens}{darkblue}  % Couleur des liens


%\usetheme[secheader]{Boadilla}   %Thème Boadilla
\usetheme{Pittsburgh}   % Thème général
\usefonttheme{default}  % Thème de polices
\setbeamertemplate{navigation symbols}{}  % Pas de menu de navigation
\setbeamertemplate{itemize item}[x]   % Puces des listes

%option de définition du thème

%\usetheme[secheader]{Boadilla}   %Thème Boadilla

\usecolortheme{rose}
\useinnertheme[shadow]{rounded}
\usecolortheme{dolphin}
\useoutertheme{infolines}


\definecolor{lightgray}{rgb}{0.8,0.8,0.8}
\definecolor{lightgrey}{rgb}{0.8,0.8,0.8}

\definecolor{lightred}{rgb}{1,0.8,0.8}
\definecolor{lightgreen}{rgb}{0.7,1,0.7}
\definecolor{darkgreen}{rgb}{0,0.5,0}
\definecolor{darkblue}{rgb}{0,0,0.5}
\definecolor{darkyellow}{rgb}{0.5,0.5,0}
\definecolor{lightyellow}{rgb}{1,1,0.6}
\definecolor{darkcyan}{rgb}{0,0.6,0.6}
\definecolor{lightcyan}{rgb}{0.6,1,1}
\definecolor{darkorange}{rgb}{0.8,0.2,0}
\definecolor{notsodarkred}{rgb}{0.8,0,0}

\definecolor{notsodarkgreen}{rgb}{0,0.7,0}

%\definecolor{coloract}{rgb}{0,1,0}
%\definecolor{colorinh}{rgb}{1,0,0}
\colorlet{coloract}{darkgreen}
\colorlet{colorinh}{red}
\colorlet{coloractgray}{lightgreen}
\colorlet{colorinhgray}{lightred}
\colorlet{colorinf}{darkgray}
\colorlet{coloractgray}{lightgreen}
\colorlet{colorinhgray}{lightred}

\colorlet{colorgray}{lightgray}
\colorlet{colorhl}{blue}



\colorlet{colorb}{blue}
\colorlet{colora1}{yellow}
\colorlet{colora0}{green}
\colorlet{colora1font}{darkyellow}
\colorlet{colora0font}{darkgreen}

\colorlet{exanswer}{blue}
\colorlet{colorgray}{lightgray}

\definecolor{colortitle}{rgb}{0.54,0.8,0.9}




\setbeamerfont{frametitle}{size=\Large}  % Police des titres


% Flèche grise
\newcommand{\fg}{\textcolor{couleurtheme}{\textbf{$\rightarrow$\ }}}
\newcommand{\FG}{\textcolor{couleurtheme}{\textbf{$\Rightarrow$\ }}}

% Environnement liste avec flèches
\newenvironment{fleches}{%
\begin{list}{}{%
\setlength{\labelwidth}{1em}% largeur de la boîte englobant le label
\setlength{\labelsep}{0pt}% espace entre paragraphe et l’étiquette
%\setlength{\itemsep}{1pt}
%\setlength{\leftmargin}{\labelwidth+\labelsep}% marge de gauche
\renewcommand{\makelabel}{\fg}%
}}{\end{list}}

% Liste sans puce
\newenvironment{liste}{%
\begin{list}{}{%
\setlength{\labelwidth}{0em}% largeur de la boîte englobant le label
\setlength{\labelsep}{0pt}% espace entre paragraphe et l’étiquette
\setlength{\leftmargin}{0em}% marge de gauche
%\renewcommand{\makelabel}{\fg}%
}}{\end{list}}

% Style des exemples
\newcommand{\ex}[1]{\textcolor{couleurex}{#1}}
\newcommand{\qex}[1]{\quad \ex{#1}}
\newcommand{\rex}[1]{\hfill \ex{#1}}
\newcommand{\redex}[1]{\textcolor{couleurredex}{#1}}

\newcommand{\lien}[1]{\textcolor{couleurliens}{\underline{\url{#1}}}}

%\newcommand{\console}[1]{\textcolor{darkgray}{#1}}

% % Style des citations
% \newcommand{\tscite}[1]{\textcolor{couleurcit}{#1}}
% \newcommand{\tcite}[1]{\textcolor{couleurcit}{[#1]}}
% \newcommand{\tcitebullet}{~~$\textcolor{couleurcit}{\bullet}$~}

% Style des citations
\newcommand{\tscite}[1]{\textcolor{couleurcit}{#1}}
\newcommand{\tcite}[1]{\tscite{[#1]}}
\newcommand{\tcitesm}[1]{{\small{\tcite{#1}}}}
\newcommand{\tcitebullet}{~~$\textcolor{couleurtheme}{\bullet}$~}
\newcommand{\simplebullet}{$\textcolor{couleurtheme}{\bullet}$~}



% Style de texte mis en valeur
\newcommand{\tval}[1]{\textcolor{couleurex}{#1}}%{\textbf{#1}}

% Un vrai symbole pour l'ensemble vide
\renewcommand{\emptyset}{\varnothing}

% Pour définir la conférence et son nom court
\newcommand{\conference}[2]{\def\theconference{#2}
\def\insertshortconference{\ifthenelse{\equal{#1}{-}}{#2}{\ifthenelse{\equal{#1}{}}{#2}{#1}}}}



\newcommand{\thedate}{09/03/2017}
\date{\thedate}
\conference{LIPN seminar}{LIPN seminar}
\title[]{Hybrid Modelling, Analysis and  Verification of large-scale Biological Networks}
\author{Louis FIPPO FITIME}




\setbeamercovered{transparent=0}
\setbeamertemplate{footline}{\color{couleurtheme}%
\scriptsize
\quad\strut%
\insertauthor%
\hfill%
\insertframenumber/\inserttotalframenumber%
\hfill%
\insertshortconference{} --- \thedate\quad\strut
}


\newcommand{\headersep}{$\circ$} % \bullet \triangleright

\setbeamertemplate{headline}{\color{couleurtheme}%
\vskip0.3em%
\quad\strut%
{\scriptsize\color{black}%
% Gris si une section existe
\ifthenelse{\equal{\thesection}{0}}{}{%
\ifthenelse{\equal{\lastsection}{x}}{}{%
\color{couleurtheme}%
}}%
\insertshorttitle
\ifthenelse{\equal{\thesection}{0}}{}{%
\ifthenelse{\equal{\lastsection}{x}}{}{%
~\headersep{} %
% Gris si une sous-section existe
\ifthenelse{\equal{\thesubsection}{0}}{\color{black}}{%
\ifthenelse{\equal{\lastsubsection}{x}}{\color{black}}{%
\color{couleurtheme}%
}}%
\insertsectionhead%
%
\ifthenelse{\equal{\thesubsection}{0}}{}{%
\ifthenelse{\equal{\lastsubsection}{x}}{}{%
~\headersep{} \color{black}\insertsubsectionhead%
%
}}}}}%
\vskip-5ex%
}



\def \scaleex {0.80}
\def \scaleminiex {0.6}
\def \scaleinf {0.6}


