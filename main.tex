\documentclass[fleqn,10pt,c]{beamer}

\usepackage[english]{babel}
\usepackage[utf8]{inputenc}
\usepackage[T1]{fontenc}


\usepackage{amsmath} % Maths
\usepackage{amsfonts} % Maths
\usepackage{amssymb} % Maths
\usepackage{stmaryrd} % Maths (crochets doubles)

%\usepackage{listings} % Mise en forme du code (pour Hoare) ## À REVOIR ###
%\usepackage{ifthen} % Structures If Then Else
\usepackage{theorem} % Styles supplémentaires pour théorèmes
\usepackage{url}
\usepackage{array}  % Tableaux évolués
\usepackage{multirow}  % Pour des colonnes sur plusieurs lignes
%\usepackage[svgnames]{xcolor} %pour les couleurs de svgnames
\usepackage{color} %pour les couleur

%\usepackage{enumerate} % Changer les puces des listes d'énumération
%\usepackage{setspace} % Changer les interlignes

%\usepackage{subfig} % Créer des sous-figures
%\usepackage{graphicx} % Importer des images

\usepackage{ulem}  % Pour l'attribut barré

\usepackage{comment}

% Police
\usepackage{lmodern}
%\usepackage{libertine}

\usepackage{tikz}
\usepackage{tkz-tab}
\newdimen\pgfex
\newdimen\pgfem
%\usetikzlibrary{arrows.meta,shapes,shadows,scopes}
\usetikzlibrary{positioning}
\usetikzlibrary{matrix}
\usetikzlibrary{decorations.text}
\usetikzlibrary{decorations.pathmorphing}

% Define layers
\pgfdeclarelayer{background}
\pgfdeclarelayer{foreground}
\pgfsetlayers{background,main,foreground}


%il faudra rajouter des macros

% Commande À FAIRE
\usepackage{color} % Couleurs du texte
%\newcommand{\afaire}[1]{\textcolor{red}{[À FAIRE : #1]}}
%\newcommand{\todo}[1]{\textcolor{red}{\textbf{[TODO\ifthenelse{\equal{#1}{}}{}{: #1}]}}}



\colorlet{couleurtheme}{gray}  % Couleur principale du thème
\colorlet{couleurcit}{darkcyan}  % Couleur des citations
\colorlet{couleurex}{blue}  % Couleur des exemples
\colorlet{couleurredex}{red}  % Couleur des exemples
\colorlet{couleurliens}{darkblue}  % Couleur des liens


%\usetheme[secheader]{Boadilla}   %Thème Boadilla
\usetheme{Pittsburgh}   % Thème général
\usefonttheme{default}  % Thème de polices
\setbeamertemplate{navigation symbols}{}  % Pas de menu de navigation
\setbeamertemplate{itemize item}[x]   % Puces des listes

%option de définition du thème

%\usetheme[secheader]{Boadilla}   %Thème Boadilla

\usecolortheme{rose}
\useinnertheme[shadow]{rounded}
\usecolortheme{dolphin}
\useoutertheme{infolines}


\definecolor{lightgray}{rgb}{0.8,0.8,0.8}
\definecolor{lightgrey}{rgb}{0.8,0.8,0.8}

\definecolor{lightred}{rgb}{1,0.8,0.8}
\definecolor{lightgreen}{rgb}{0.7,1,0.7}
\definecolor{darkgreen}{rgb}{0,0.5,0}
\definecolor{darkblue}{rgb}{0,0,0.5}
\definecolor{darkyellow}{rgb}{0.5,0.5,0}
\definecolor{lightyellow}{rgb}{1,1,0.6}
\definecolor{darkcyan}{rgb}{0,0.6,0.6}
\definecolor{lightcyan}{rgb}{0.6,1,1}
\definecolor{darkorange}{rgb}{0.8,0.2,0}
\definecolor{notsodarkred}{rgb}{0.8,0,0}

\definecolor{notsodarkgreen}{rgb}{0,0.7,0}

%\definecolor{coloract}{rgb}{0,1,0}
%\definecolor{colorinh}{rgb}{1,0,0}
\colorlet{coloract}{darkgreen}
\colorlet{colorinh}{red}
\colorlet{coloractgray}{lightgreen}
\colorlet{colorinhgray}{lightred}
\colorlet{colorinf}{darkgray}
\colorlet{coloractgray}{lightgreen}
\colorlet{colorinhgray}{lightred}

\colorlet{colorgray}{lightgray}
\colorlet{colorhl}{blue}



\colorlet{colorb}{blue}
\colorlet{colora1}{yellow}
\colorlet{colora0}{green}
\colorlet{colora1font}{darkyellow}
\colorlet{colora0font}{darkgreen}

\colorlet{exanswer}{blue}
\colorlet{colorgray}{lightgray}

\definecolor{colortitle}{rgb}{0.54,0.8,0.9}




\setbeamerfont{frametitle}{size=\Large}  % Police des titres


% Flèche grise
\newcommand{\fg}{\textcolor{couleurtheme}{\textbf{$\rightarrow$\ }}}
\newcommand{\FG}{\textcolor{couleurtheme}{\textbf{$\Rightarrow$\ }}}

% Environnement liste avec flèches
\newenvironment{fleches}{%
\begin{list}{}{%
\setlength{\labelwidth}{1em}% largeur de la boîte englobant le label
\setlength{\labelsep}{0pt}% espace entre paragraphe et l’étiquette
%\setlength{\itemsep}{1pt}
%\setlength{\leftmargin}{\labelwidth+\labelsep}% marge de gauche
\renewcommand{\makelabel}{\fg}%
}}{\end{list}}

% Liste sans puce
\newenvironment{liste}{%
\begin{list}{}{%
\setlength{\labelwidth}{0em}% largeur de la boîte englobant le label
\setlength{\labelsep}{0pt}% espace entre paragraphe et l’étiquette
\setlength{\leftmargin}{0em}% marge de gauche
%\renewcommand{\makelabel}{\fg}%
}}{\end{list}}

% Style des exemples
\newcommand{\ex}[1]{\textcolor{couleurex}{#1}}
\newcommand{\qex}[1]{\quad \ex{#1}}
\newcommand{\rex}[1]{\hfill \ex{#1}}
\newcommand{\redex}[1]{\textcolor{couleurredex}{#1}}

\newcommand{\lien}[1]{\textcolor{couleurliens}{\underline{\url{#1}}}}

%\newcommand{\console}[1]{\textcolor{darkgray}{#1}}

% % Style des citations
% \newcommand{\tscite}[1]{\textcolor{couleurcit}{#1}}
% \newcommand{\tcite}[1]{\textcolor{couleurcit}{[#1]}}
% \newcommand{\tcitebullet}{~~$\textcolor{couleurcit}{\bullet}$~}

% Style des citations
\newcommand{\tscite}[1]{\textcolor{couleurcit}{#1}}
\newcommand{\tcite}[1]{\tscite{[#1]}}
\newcommand{\tcitesm}[1]{{\small{\tcite{#1}}}}
\newcommand{\tcitebullet}{~~$\textcolor{couleurtheme}{\bullet}$~}
\newcommand{\simplebullet}{$\textcolor{couleurtheme}{\bullet}$~}



% Style de texte mis en valeur
\newcommand{\tval}[1]{\textcolor{couleurex}{#1}}%{\textbf{#1}}

% Un vrai symbole pour l'ensemble vide
\renewcommand{\emptyset}{\varnothing}

% Pour définir la conférence et son nom court
\newcommand{\conference}[2]{\def\theconference{#2}
\def\insertshortconference{\ifthenelse{\equal{#1}{-}}{#2}{\ifthenelse{\equal{#1}{}}{#2}{#1}}}}



\newcommand{\thedate}{09/03/2017}
\date{\thedate}
\conference{LIPN seminar}{LIPN seminar}
\title[]{Hybrid Modelling, Analysis and  Verification of large-scale Biological Networks}
\author{Louis FIPPO FITIME}




\setbeamercovered{transparent=0}
\setbeamertemplate{footline}{\color{couleurtheme}%
\scriptsize
\quad\strut%
\insertauthor%
\hfill%
\insertframenumber/\inserttotalframenumber%
\hfill%
\insertshortconference{} --- \thedate\quad\strut
}


\newcommand{\headersep}{$\circ$} % \bullet \triangleright

\setbeamertemplate{headline}{\color{couleurtheme}%
\vskip0.3em%
\quad\strut%
{\scriptsize\color{black}%
% Gris si une section existe
\ifthenelse{\equal{\thesection}{0}}{}{%
\ifthenelse{\equal{\lastsection}{x}}{}{%
\color{couleurtheme}%
}}%
\insertshorttitle
\ifthenelse{\equal{\thesection}{0}}{}{%
\ifthenelse{\equal{\lastsection}{x}}{}{%
~\headersep{} %
% Gris si une sous-section existe
\ifthenelse{\equal{\thesubsection}{0}}{\color{black}}{%
\ifthenelse{\equal{\lastsubsection}{x}}{\color{black}}{%
\color{couleurtheme}%
}}%
\insertsectionhead%
%
\ifthenelse{\equal{\thesubsection}{0}}{}{%
\ifthenelse{\equal{\lastsubsection}{x}}{}{%
~\headersep{} \color{black}\insertsubsectionhead%
%
}}}}}%
\vskip-5ex%
}



\def \scaleex {0.80}
\def \scaleminiex {0.6}
\def \scaleinf {0.6}



\usepackage{ifthen}

\newcommand{\currentScope}{}
\newcommand{\currentSort}{}
\newcommand{\currentSortLabel}{}
\newcommand{\currentAlign}{}
\newcommand{\currentSize}{}

\newcounter{la}
\newcommand{\TSetSortLabel}[2]{
  \expandafter\repcommand\expandafter{\csname TUserSort@#1\endcsname}{#2}
}
\newcommand{\TSort}[4]{
  \renewcommand{\currentScope}{#1}
  \renewcommand{\currentSort}{#2}
  \renewcommand{\currentSize}{#3}
  \renewcommand{\currentAlign}{#4}
  \ifcsname TUserSort@\currentSort\endcsname
    \renewcommand{\currentSortLabel}{\csname TUserSort@\currentSort\endcsname}
  \else
    \renewcommand{\currentSortLabel}{\currentSort}
  \fi
  \begin{scope}[shift={\currentScope}]
  \ifthenelse{\equal{\currentAlign}{l}}{
    \filldraw[process box] (-0.5,-0.5) rectangle (0.5,\currentSize-0.5);
    \node[sort] at (-0.2,\currentSize-0.4) {\currentSortLabel};
   }{\ifthenelse{\equal{\currentAlign}{r}}{
     \filldraw[process box] (-0.5,-0.5) rectangle (0.5,\currentSize-0.5);
     \node[sort] at (0.2,\currentSize-0.4) {\currentSortLabel};
   }{
    \filldraw[process box] (-0.5,-0.5) rectangle (\currentSize-0.5,0.5);
    \ifthenelse{\equal{\currentAlign}{t}}{
      \node[sort,anchor=east] at (-0.3,0.2) {\currentSortLabel};
    }{
      \node[sort] at (-0.6,-0.2) {\currentSortLabel};
    }
   }}
  \setcounter{la}{\currentSize}
  \addtocounter{la}{-1}
  \foreach \i in {0,...,\value{la}} {
    \TProc{\i}
  }
  \end{scope}
}

\newcommand{\TTickProc}[2]{ % pos, label
  \ifthenelse{\equal{\currentAlign}{l}}{
    \draw[tick] (-0.6,#1) -- (-0.4,#1);
    \node[tick label, anchor=east] at (-0.55,#1) {#2};
   }{\ifthenelse{\equal{\currentAlign}{r}}{
    \draw[tick] (0.6,#1) -- (0.4,#1);
    \node[tick label, anchor=west] at (0.55,#1) {#2};
   }{
    \ifthenelse{\equal{\currentAlign}{t}}{
      \draw[tick] (#1,0.6) -- (#1,0.4);
      \node[tick label, anchor=south] at (#1,0.55) {#2};
    }{
      \draw[tick] (#1,-0.6) -- (#1,-0.4);
      \node[tick label, anchor=north] at (#1,-0.55) {#2};
    }
   }}
}
\newcommand{\TSetTick}[3]{
  \expandafter\repcommand\expandafter{\csname TUserTick@#1_#2\endcsname}{#3}
}

\newcommand{\myProc}[3]{
  \ifcsname TUserTick@\currentSort_#1\endcsname
    \TTickProc{#1}{\csname TUserTick@\currentSort_#1\endcsname}
  \else
    \TTickProc{#1}{#1}
  \fi
  \ifthenelse{\equal{\currentAlign}{l}\or\equal{\currentAlign}{r}}{
    \node[#2] (\currentSort_#1) at (0,#1) {#3};
  }{
    \node[#2] (\currentSort_#1) at (#1,0) {#3};
  }
}
\newcommand{\TSetProcStyle}[2]{
  \expandafter\repcommand\expandafter{\csname TUserProcStyle@#1\endcsname}{#2}
}
\newcommand{\TProc}[1]{
  \ifcsname TUserProcStyle@\currentSort_#1\endcsname
    \myProc{#1}{\csname TUserProcStyle@\currentSort_#1\endcsname}{}
  \else
    \myProc{#1}{process}{}
  \fi
}

\newcommand{\repcommand}[2]{
  \providecommand{#1}{#2}
  \renewcommand{#1}{#2}
}
\newcommand{\THit}[5]{
  \path[hit] (#1) edge[#2] (#3#4);
  \expandafter\repcommand\expandafter{\csname TBounce@#3@#5\endcsname}{#4}
}
\newcommand{\TBounce}[4]{
  (#1\csname TBounce@#1@#3\endcsname) edge[#2] (#3#4)
}

%\newcommand{\TState}[1]{
%  \foreach \proc in {#1} {
%    \node[current process] (\proc) at (\proc.center) {};
%  }
%}

\newcommand{\TState}[2]{
  \foreach \proc in {#2} {
        \only<#1>{ \node[current process] (\proc) at (\proc.center) {}; }
  };
}


%aujout des styles tikz pour dessiner les frappes de processus et les réseaux d'automates
\tikzstyle{sort}=[fill=lightgray, rounded corners, draw=black]
\tikzstyle{process}=[circle,draw,minimum size=15pt,font=\footnotesize,inner sep=1pt]
\tikzstyle{black process}=[process, draw=blue, fill=red,text=black,font=\bfseries]
\tikzstyle{highlighted process}=[current process, fill=gray]
\tikzstyle{process box}=[fill=none,draw=black,rounded corners]
\tikzstyle{current process}=[process,fill=blue]
\tikzstyle{tick label}=[font=\footnotesize]
\tikzstyle{tick}=[densely dotted]
\tikzstyle{hit}=[->,>=angle 45]
\tikzstyle{selfhit}=[min distance=30pt,curve to]
\tikzstyle{bounce}=[densely dotted,>=stealth',->]
\tikzstyle{hlhit}=[very thick]
\tikzstyle{ulhit}=[draw=lightgray,fill=lightgray]
\tikzstyle{pulhit}=[fill=lightgray]
\tikzstyle{bulhit}=[draw=lightgray]
\tikzstyle{rate}=[]

\tikzstyle{hitless graph}=[every edge/.style={draw=red,-}]

\tikzstyle{aS}=[every edge/.style={draw,->}]
\tikzstyle{Asol}=[draw,circle,minimum size=5pt,inner sep=0]
\tikzstyle{Aproc}=[draw]
\tikzstyle{Aobj}=[]


%pour dessiner le graphe d'états
\tikzstyle{arc0}=[->]
\tikzstyle{nd0}=[]
\tikzstyle{arc1}=[->]
\tikzstyle{nd1}=[]
\tikzstyle{arc2}=[->]
\tikzstyle{nd2}=[]
\tikzstyle{arc3}=[->]
\tikzstyle{nd3}=[]


\tikzstyle{boxed ph}=[]
\tikzstyle{sort}=[fill=lightgray, rounded corners, draw=black]
\tikzstyle{process}=[circle,draw,minimum size=15pt,fill=white,font=\footnotesize,inner sep=1pt]
%\tikzstyle{black process}=[process, draw=blue, fill=red,text=black,font=\bfseries]
\tikzstyle{gray process}=[process, draw=black, fill=lightgray]
\tikzstyle{highlighted process}=[current process, fill=gray]
\tikzstyle{process box}=[fill=none,draw=black,rounded corners]
%\tikzstyle{current process}=[process, draw=black, fill=lightgray]
\tikzstyle{current process}=[process,fill=blue]
\tikzstyle{hl process}=[process,fill=blue!30]
\tikzstyle{tick label}=[font=\footnotesize]
\tikzstyle{tick}=[densely dotted] %-
\tikzstyle{hit}=[->,>=angle 45]
\tikzstyle{selfhit}=[min distance=50pt,curve to]
\tikzstyle{bounce}=[densely dotted,>=stealth',->]
\tikzstyle{ulhit}=[draw=lightgray,fill=lightgray]
\tikzstyle{pulhit}=[fill=lightgray]
\tikzstyle{bulhit}=[draw=lightgray]
\tikzstyle{hl}=[very thick,colorhl]
\tikzstyle{hlb}=[very thick]
\tikzstyle{hlhit}=[hl]

\tikzstyle{update}=[draw,->,dashed,shorten >=.7cm,shorten <=.7cm]

\tikzstyle{unprio}=[draw,thin]%[double]
%\tikzstyle{prio}=[draw,thick,-stealth]%[double]
\tikzstyle{prio}=[draw,-stealth,double]

\tikzstyle{hitless graph}=[every edge/.style={draw=red,-}]

\tikzstyle{aS}=[every edge/.style={draw,->,>=stealth}]
\tikzstyle{Asol}=[draw,circle,minimum size=5pt,inner sep=0,node distance=1cm]
\tikzstyle{Aproc}=[draw,node distance=1.2cm]
\tikzstyle{Aobj}=[node distance=1.5cm]
\tikzstyle{Anos}=[font=\Large]

\tikzstyle{AsolPrio}=[Asol,double]
\tikzstyle{AprocPrio}=[Aproc,double]
\tikzstyle{aSPrio}=[aS,double]

\colorlet{colorhlwarn}{notsodarkred}
\colorlet{colorhlwarnbg}{lightred}
\tikzstyle{Ahl}=[very thick,fill=colorhlwarnbg,draw=colorhlwarn,text=colorhlwarn]
\tikzstyle{Ahledge}=[very thick,double=colorhlwarnbg,draw=colorhlwarn,color=colorhlwarn]
\tikzstyle{Aex}=[thick,draw=couleurex,fill=couleurex!20]
\tikzstyle{Aexedge}=[very thick,draw=couleurex,color=couleurex]


% Figure de résumé des liens entre les formalismes
\tikzstyle{equiv-externe}=[thick, rounded corners, draw=gray, fill=gray!10, align=center,
  inner sep=8]
  
\tikzstyle{local transitions}=[->,>=latex',thick,bend left=30,
                 		every node/.style={fill=white,inner sep=1pt,outer sep=1pt}]
\tikzstyle{reach}=[fill=lightgray,ellipse]

\tikzstyle{adn}=[every node/.style={circle,draw=black,outer sep=2pt,minimum
                size=15pt,text=black}, node distance=1.5cm, ->]
\tikzstyle{elabel}=[fill=none,text=black, above=-2pt,%sloped,
minimum size=10pt, outer sep=0, font=\scriptsize, draw=none]
\tikzstyle{labelproba}=[node distance=1.1cm,color=lightblue]




\tikzstyle{grn}=[every node/.style={circle,draw=black,outer sep=2pt,minimum
                size=15pt,text=black}, node distance=1.5cm]
\tikzstyle{inh}=[>=|,-|,draw=colorinh]%,thick, text=black,label]
\tikzstyle{act}=[->,>=triangle 60,draw=coloract,thick,color=coloract]
\tikzstyle{inhgray}=[>=|,-|,draw=colorinhgray,thick, text=black,label]
\tikzstyle{actgray}=[->,>=triangle 60,draw=coloractgray,thick,color=coloractgray]
\tikzstyle{inf}=[->,draw=colorinf,thick,color=colorinf]
%\tikzstyle{elabel}=[fill=none, above=-1pt, sloped,text=black, minimum size=10pt, outer sep=0, font=\scriptsize,draw=none]
\tikzstyle{elabel}=[fill=none,text=black, above=-2pt,%sloped,
minimum size=10pt, outer sep=0, font=\scriptsize, draw=none]
\tikzstyle{reach}=[->,dashed]


\tikzstyle{plot}=[every path/.style={-}]
\tikzstyle{axe}=[gray,->,>=stealth']
\tikzstyle{ticks}=[font=\scriptsize,every node/.style={gray}]
\tikzstyle{mean}=[thick]
\tikzstyle{interval}=[line width=5pt,red,draw opacity=0.7]

\tikzstyle{hl}=[yellow]
\tikzstyle{hl2}=[orange]

\tikzstyle{every matrix}=[ampersand replacement=\&]
\tikzstyle{shorthandoff}=[]
\tikzstyle{shorthandon}=[]


%%%%%%%%%%%%%%%%%%%%%%%%%%%%%%%%%%%%%%%%%%%%%%%%%%%%%%%%%%%%%%%%%%%%%%%%%%%%%%%%%%%%%%%%%%%
%%%%%%%%%%%%%%%%%%Definition des éléments pour un reseau de signalisation%%%%%%%%%%%%%%%%%
\tikzstyle{sn}=[circle, draw, thin,fill=cyan!20, scale=0.8] %seed node
\tikzstyle{snpat}=[circle, draw=purple, thin, scale=0.8] %identify patterns
\tikzstyle{ps}=[rectangle, draw, thin,fill=white!20, scale=0.8] %protein de signalisation
\tikzstyle{cplx}=[square, draw, thin, fill=white!20, scale=0.8] %définition d'un complex
\tikzstyle{ecad}=[square, draw, thin, fill=blue!20, scale=0.8] %définition de l'input
\tikzstyle{transl}=[diamond, draw, thin, fill=white!20, scale=0.3]
\tikzstyle{mod}=[triangle, draw, thin, fill=white!20, scale=0.3]
\tikzstyle{qgre}=[rectangle, draw, thin,fill=green!20, size=15pts]
\tikzstyle{tgrn}=[triangle, draw, thin, fill=green!20, scale=0.8]

%Ajout des arc qui ne sont pas définis dans le grn
\tikzstyle{st}=[->, draw, thin, dashed] %définition d'un state transition
\tikzstyle{stv}=[->, draw=purple, thin, dashed, ultra thick]
\tikzstyle{inhN}=[>=|,-|,draw=colorinh,thick, text=black,label,scale=0.1]
\tikzstyle{actN}=[->,>=triangle 60,draw=coloract,thick,color=coloract,scale=0.1]






%\definecolor{darkred}{rgb}{0.5,0,0}
\definecolor{lightred}{rgb}{1,0.8,0.8}
\definecolor{lightgreen}{rgb}{0.7,1,0.7}
\definecolor{darkgreen}{rgb}{0,0.5,0}
\definecolor{darkblue}{rgb}{0,0,0.5}
\definecolor{darkyellow}{rgb}{0.5,0.5,0}
\definecolor{lightyellow}{rgb}{1,1,0.6}
\definecolor{darkcyan}{rgb}{0,0.6,0.6}
\definecolor{darkorange}{rgb}{0.8,0.2,0}
\definecolor{Maroon}{cmyk}{0, 0.87, 0.68, 0.32}

\definecolor{notsodarkgreen}{rgb}{0,0.7,0}

%\definecolor{coloract}{rgb}{0,1,0}
%\definecolor{colorinh}{rgb}{1,0,0}
\colorlet{coloract}{darkgreen}
\colorlet{colorinh}{red}
\colorlet{coloractgray}{lightgreen}
\colorlet{colorinhgray}{lightred}
\colorlet{colorinf}{darkgray}
\colorlet{coloractgray}{lightgreen}
\colorlet{colorinhgray}{lightred}
\colorlet{colorgray}{lightgray}
\definecolor{lightred}{rgb}{1,0.3,0.3}
\tikzstyle{hl}=[yellow]
\tikzstyle{hl2}=[orange]
\colorlet{couleurtheme}{gray}  % Couleur principale du thème
\colorlet{couleurcit}{darkcyan}  % Couleur des citations
\colorlet{couleurex}{blue}  % Couleur des exemples
\colorlet{couleurredex}{red}  % Couleur des exemples
\colorlet{couleurliens}{darkblue}  % Couleur des liens


%\usetheme[secheader]{Boadilla}   %Thème Boadilla
\usetheme{Pittsburgh}   % Thème général
\usefonttheme{default}  % Thème de polices
\setbeamertemplate{navigation symbols}{}  % Pas de menu de navigation
\setbeamertemplate{itemize item}[x]   % Puces des listes

%option de définition du thème

%\usetheme[secheader]{Boadilla}   %Thème Boadilla

\usecolortheme{rose}
\useinnertheme[shadow]{rounded}
\usecolortheme{dolphin}
\useoutertheme{infolines}


\setbeamerfont{frametitle}{size=\Large}  % Police des titres


% Flèche grise
\newcommand{\fg}{\textcolor{couleurtheme}{\textbf{$\rightarrow$\ }}}
\newcommand{\FG}{\textcolor{couleurtheme}{\textbf{$\Rightarrow$\ }}}

% Environnement liste avec flèches
\newenvironment{fleches}{%
\begin{list}{}{%
\setlength{\labelwidth}{1em}% largeur de la boîte englobant le label
\setlength{\labelsep}{0pt}% espace entre paragraphe et l’étiquette
%\setlength{\itemsep}{1pt}
%\setlength{\leftmargin}{\labelwidth+\labelsep}% marge de gauche
\renewcommand{\makelabel}{\fg}%
}}{\end{list}}

% Liste sans puce
\newenvironment{liste}{%
\begin{list}{}{%
\setlength{\labelwidth}{0em}% largeur de la boîte englobant le label
\setlength{\labelsep}{0pt}% espace entre paragraphe et l’étiquette
\setlength{\leftmargin}{0em}% marge de gauche
%\renewcommand{\makelabel}{\fg}%
}}{\end{list}}

% Style des exemples
\newcommand{\ex}[1]{\textcolor{couleurex}{#1}}
\newcommand{\qex}[1]{\quad \ex{#1}}
\newcommand{\rex}[1]{\hfill \ex{#1}}
\newcommand{\redex}[1]{\textcolor{couleurredex}{#1}}

\newcommand{\lien}[1]{\textcolor{couleurliens}{\underline{\url{#1}}}}


% Style des citations
\newcommand{\tscite}[1]{\textcolor{couleurcit}{#1}}
\newcommand{\tcite}[1]{\tscite{[#1]}}
\newcommand{\tcitesm}[1]{{\small{\tcite{#1}}}}
\newcommand{\tcitebullet}{~~$\textcolor{couleurtheme}{\bullet}$~}
\newcommand{\simplebullet}{$\textcolor{couleurtheme}{\bullet}$~}



% Style de texte mis en valeur
\newcommand{\tval}[1]{\textcolor{couleurex}{#1}}%{\textbf{#1}}

% Un vrai symbole pour l'ensemble vide
\renewcommand{\emptyset}{\varnothing}

% Pour définir la conférence et son nom court
\newcommand{\conference}[2]{\def\theconference{#2}
\def\insertshortconference{\ifthenelse{\equal{#1}{-}}{#2}{\ifthenelse{\equal{#1}{}}{#2}{#1}}}}



\newcommand{\thedate}{13/03/2017}
\date{\thedate}
\conference{PACS Meeting}{PACS Meeting}
\title[]{Parameters synthesis by using abstract
interpretation in Parametric Stochastic Automata Networks
}
\author{Louis FIPPO FITIME}




\setbeamercovered{transparent=0}
\setbeamertemplate{footline}{\color{couleurtheme}%
\scriptsize
\quad\strut%
\insertauthor%
\hfill%
\insertframenumber/\inserttotalframenumber%
\hfill%
\insertshortconference{} --- \thedate\quad\strut
}


\newcommand{\headersep}{$\circ$} % \bullet \triangleright

\setbeamertemplate{headline}{\color{couleurtheme}%
\vskip0.3em%
\quad\strut%
{\scriptsize\color{black}%
% Gris si une section existe
\ifthenelse{\equal{\thesection}{0}}{}{%
\ifthenelse{\equal{\lastsection}{x}}{}{%
\color{couleurtheme}%
}}%
\insertshorttitle
\ifthenelse{\equal{\thesection}{0}}{}{%
\ifthenelse{\equal{\lastsection}{x}}{}{%
~\headersep{} %
% Gris si une sous-section existe
\ifthenelse{\equal{\thesubsection}{0}}{\color{black}}{%
\ifthenelse{\equal{\lastsubsection}{x}}{\color{black}}{%
\color{couleurtheme}%
}}%
\insertsectionhead%
%
\ifthenelse{\equal{\thesubsection}{0}}{}{%
\ifthenelse{\equal{\lastsubsection}{x}}{}{%
~\headersep{} \color{black}\insertsubsectionhead%
%
}}}}}%
\vskip-5ex%
}



\def \scaleex {0.80}
\def \scaleminiex {0.6}
\def \scaleinf {0.6}

\colorlet{colorb}{blue}
\colorlet{colora1}{yellow}
\colorlet{colora0}{green}
\colorlet{colora1font}{darkyellow}
\colorlet{colora0font}{darkgreen}

\colorlet{exanswer}{blue}
\colorlet{colorgray}{lightgray}

\definecolor{colortitle}{rgb}{0.54,0.8,0.9}


\begin{document}



\input{parts/ex.tex}



\begin{frame}[plain,label=title]

% Cadre de titre
\begin{center}
\vspace{1cm}
\setbeamercolor{postit}{fg=black,bg=colortitle}
\begin{beamercolorbox}[sep=0.5em]{postit}
\centering
\Large
\textbf{%
{\normalsize\theconference{}}\\~\\%
\inserttitle
}
\end{beamercolorbox}

% Auteurs et instituts
% Auteurs et instituts


\par
\medskip
\normalsize
Louis Fippo Fitime$^1$
\footnotesize

\texttt{fippofitime@lipn.univ-paris13.fr}

\url{http://www.irccyn.ec-nantes.fr/~fippofit/}



\bigskip
\textbf{Joint work with:} \\  \'Etienne Andr\'e$^1$ and Laure Petrucci$^1$ 

% Auteurs et instituts

\medskip
\footnotesize
$^1$ LCR / LIPN / CNRS (Paris, France)

\texttt{etienne.andre@lipn.fr}

\texttt{laure.petrucci@lipn.univ-paris13.fr}


\end{center}

\end{frame}


\section{Context \& Motivation}

\begin{frame}[c]
  \frametitle{Context}
%rajouter le titre (version courte) dans toute les slides
\begin{center}
  \includegraphics[scale=0.12]{images/cellule-description.jpeg}
\end{center}
\begin{center}
{\tiny \color{darkgreen}[\citelui]}
\end{center}

%\tcite{Wikipédia}

\begin{itemize}
\item Cellular processes $\Longrightarrow$ networks of biological interactions.
\item Nodes (biological components), edges (interactions).
\end{itemize}

%Cellular processes are driven by networks of biological reactions. Cells rely on the tight coordination of these pathways to achieve proper functioning.
%With the help of signaling pathway, a cell senses changes in its environnement or internal state. This information is then passed on via cascades of biochemical 
%reactions to the appropriate mechanisms which respond by modifying the metabolic and transcriptiona activities. this in turn modifies the behavior of the cell.

%Consequently, the dynamics of biopathways play a crucial role in determinig cellular functions.

%Examples: circadian rhythm, the apoptosis pathway inducing programmed cell death, cell differentiation.

%\textcolor{couleurtheme}{$\Rightarrow$} \fbox{\tval{\large The need of comprehension of biological systems}} \textcolor{couleurtheme}{$\Leftarrow$}


%\textcolor{couleurtheme}{$\Rightarrow$} \fbox{\tval{\large Allow efficient translation from Process Hitting to BRN}} \textcolor{couleurtheme}{$\Leftarrow$}

\end{frame}

\begin{frame}[c]
 \frametitle{Motivation}
  %\pause
 %figure illustrative
 \begin{tikzpicture}[auto]

\path[use as bounding box] (-0.7,-2) rectangle (3,3);

%le noeud pour les connaissances de la littérature, générales
\node[align=center] (gk) at (2,3) {\begin{tabular}{|c|}
\hline
 General knowledge  \\
 \hline
 Literature  \\
  \hline
 Hypotheses   \\
  \hline
\end{tabular}};

%\pause

%les noeud pour le réseau biologique
\node[qgre] (a) at (1,1.5) {a};
\node[mod] (i) at (2.3,1) {i};
\node[qgre] (b) at (1,0.5) {b};
\node[qgre] (c) at (3,1) {c};


\path
 (a) edge[act] node[center]{$p_1$\footnotesize ?} (i)
 (b) edge[inh] node[below]{\color{red} $p_2$\footnotesize ?} (i)
 (i) edge[st]  (c);

 %\pause

\node (deco) at (2,-0.1) {Times series data};
\node[align=center] (tsd) at (2,-1) {\begin{tabular}{|c|c|c|c|}
\hline
 Genes  & 1h & ... & 24h  \\
 \hline
 Gene $1$  &   & ...  &    \\
  \hline
  Gene $2$  &   & ...  &    \\
  \hline
\end{tabular}};


\onslide<2->{

\node (d1) at (5,1) {};
\node (d2) at (7.5,1) {};

\node (d3) at (4.5,-1.5) {};
\node (d4) at (4.5,3.5) {};


\draw[->,line width=6pt, color=lightgray] (d1) -- (d2) node[above=10pt,midway]{\textcolor{black}{\textbf{Algebraic Modelling}}};
}

\onslide<2->{
%le modèle en process hitting
\node[scale=0.4] (phmodel) at (10,1) {\begin{tikzpicture} \exphHM
                           \end{tikzpicture}};
}

\end{tikzpicture}

\onslide<2->{
Formal inference of the parameters of Biological Networks:

  \begin{itemize}
   \item Avoid critical behaviors.
  \end{itemize}

}

\end{frame}
                   





\section{Parametric Stochastic Automata Networks}

\input{parts/psan.tex}



\section{Parameters estimation from time-series data}

\begin{frame}[c]
\frametitle{Approach}
\begin{tikzpicture}[grn]

 % placement des noeuds
 
  \node[block,align=center] (rstc) at (-2,5) {\begin{tabular}{c}
                                            \textbf{Biological network} \\ \hline
                                            RSTC network
                                           \end{tabular}};
  
  \node[block] (data) at (4,5) {\begin{tabular}{c}
                                            \textbf{Experimental Data} \\ \hline
                                            Time series data(TSD)
                                           \end{tabular}};
  
 \node[instruct,align=center] (phmodel) at (-2,2.5) {\begin{tabular}{c}
                                            \textbf{Hybrid Model} \\ \hline
                                            AN Model
                                           \end{tabular}};
                                           
  \node[instruct] (estimation) at (2.5,2.5) {\begin{tabular}{c}
                                            \textbf{Estimation of parameters}\\ \hline
                                            $r$ and $sa$ 
                                           \end{tabular}};
                                           
   
 % \node[instruct] (discretization) at (6.5,2.5) {\begin{tabular}{c}
  %                                          \textbf{Discretization} \\ \hline
   %                                         TSD
    %                                       \end{tabular}};
                                           
  \node[instruct] (simulation) at (-2,0) {\textbf{Simulation}};
    
  
  \node[test] (test) at (3,0) {Model validation};
  
 %placement des arrêtes
  \draw[suite] (rstc) --node[inner,left]{1} (phmodel);
  \draw[suite] (phmodel) --node[inner,left]{4} (simulation);
  \draw[suite] (simulation) --node[inner,above]{5} (test.west);
  
  \draw[suite] (data) --node[inner,left]{2} (estimation);
  \draw[suite] (estimation) --node[inner,above]{3} (phmodel);
  %\draw[suite] (data) --node[inner,left,node distance=5mm]{2'} (discretization);
  %\draw[suite] (discretization) --node[inner,left,node distance=1cm]{5} (test.east);
\end{tikzpicture}

\end{frame}


\begin{frame}[c]
 \frametitle{RSTC Network}
 \framesubtitle{multi-layer receptor-signaling-transcription-cell state}
%\begin{beamer}

\begin{center}
  \includegraphics[scale=0.07]{figs/net.jpg}
\end{center}
 
%\end{beamer}

%\pause

\begin{itemize}
 \item Pathway Interaction Database
 \item \tval{$293$  nodes}: signaling proteins, transcription factors, mRNA expressions
 \item \tval{$375$  interactions}: activations, inhibitions, complexes dissociation
\end{itemize}

 
\end{frame}

\begin{frame}[c]
 \frametitle{RSTC Network}
 \framesubtitle{multi-layer receptor-signaling-transcription-cell state}

%%%image zoomée
\begin{tikzpicture}[node distance = 1em,dashed,red,thick]
	\zoomZero{\includegraphics[width=0.35\textwidth]{figs/net.jpg}}
	\zoomIn[right]{\includegraphics[scale=0.15]{figs/netzoom.png}}{0.35,0.005}{0.320,0.420}
	%\zoomIn{\includegraphics[width=0.45\textwidth]{Geant3}}{0.45,0.475}{0.120,0.120}
	%\zoomIn[left]{\includegraphics[width=0.45\textwidth]{Geant2}}{0.45,0.475}{0.120,0.120}
	%\zoomIn{\includegraphics[width=0.45\textwidth]{Geant1}}{0.45,0.475}{0.120,0.120}
	%\zoomIn[right]{\includegraphics[width=0.45\textwidth]{Geant0}}{0.45,0.475}{0.120,0.120}
\end{tikzpicture}


\end{frame}

\begin{frame}[c]
 \frametitle{RSTC Network}
 \framesubtitle{multi-layer receptor-signaling-transcription-cell state}

%%%image zoomée
\begin{tikzpicture}[node distance = 1em,dashed,red,thick]
	\zoomZero{\includegraphics[width=0.45\textwidth]{figs/net.jpg}}
	\zoomIn[left]{\includegraphics[scale=0.25]{figs/netzoom.png}}{0.35,0.005}{0.320,0.420}
	%\zoomIn{\includegraphics[width=0.45\textwidth]{Geant3}}{0.45,0.475}{0.120,0.120}
	%\zoomIn[left]{\includegraphics[width=0.45\textwidth]{Geant2}}{0.45,0.475}{0.120,0.120}
	%\zoomIn{\includegraphics[width=0.45\textwidth]{Geant1}}{0.45,0.475}{0.120,0.120}
	%\zoomIn[right]{\includegraphics[width=0.45\textwidth]{Geant0}}{0.45,0.475}{0.120,0.120}
\end{tikzpicture}


\end{frame}


\begin{frame}[c]
  \frametitle{Time series data}
  
\begin{center}
  \includegraphics[width=70mm]{figs/12genes.png}
\end{center}

%\pause

\begin{columns}
\begin{column}{0.7\textwidth}
\begin{itemize}
  \item Experiment: \tval{calcium stimuli}
  \item Measured at 10 time-points(0-24hrs)
  \item \tval{$200$ transcripts} selected  (\tval{dynamic patterns}) %their fold expression with respect to the non-stimulated cell was significant in at least one time point
  \item We included in our model a subset of $12$ of them
\end{itemize}
\end{column}

\begin{column}{0.3\textwidth}
% \textbf{ \small Prof. Dr. Peter Angel
%Signal Transduction and Growth Control (A100)
%German Cancer research center
%Heidelberg, Germany}

\end{column}
\end{columns}
%\textcolor{couleurtheme}{$\Rightarrow$} \fbox{\tval{\large Allow efficient translation from Process Hitting to BRN}} \textcolor{couleurtheme}{$\Leftarrow$}

\end{frame}



\begin{frame}[c]
 \frametitle{Data estimation from TSD}
% \framesubtitle{Parameters inference/Principe}
 
\begin{columns}

\begin{column}{0.7\textwidth}
\scalebox{0.9}{
\begin{tikzpicture}[scale = 0.8]
    % Tracé de la parabole
    %\draw[red, domain = -2.2:2.2, smooth] plot (\x, {(\x)^2});
    % Alternative par Gnuplot
    %\draw[red, domain = -2.2:2.2, smooth] plot function{x**2};
    % Lignes tiretées
      \draw[thick, ->] (-.2,0)--(11,0) node[below]{$t$};
     \foreach \t in {0,1,2,3,4,...,10}
      \draw[very thick] (\t,2pt)--(\t,-2pt) node[below,blue]{\small\t};

     \draw[thick, ->] (0,-.2)--(0,4) node[left]{$b$};
     \foreach \y in {1,2,3}
      \draw[very thick] (2pt,\y)--(-2pt,\y) node[left,blue]{\small\y};
      
    \draw[thick] plot[mark=ball,mark size=1pt] file {illustration.txt};
    
    \onslide<2->{
    \draw[thick,|<->|] (2.8,3.5) -- (3.2,3.5) node[above,blue]{$Max$};
    \draw[thick,|<->|] (2.8,-.2) -- (3.2,-.2) node[below,blue]{$Min$};
    }
    \onslide<3->{
    \draw[thick,|<->|] (3,0) -- (3,3.5);
    }
    \onslide<4->{
    \draw[thick,dotted,blue] (0,1.16) -- (10,1.16) node[right]{$th1$}; 
    \draw[thick,dotted,blue] (0,2.33) -- (10,2.33) node[right]{$th2$}; 
    }
    \onslide<5->{
    \draw[thick,dotted,purple] (0.7,1.16) -- (0.7,-.5) node[below]{$t_{1}$};
    }
    \onslide<6->{
    \draw[thick,dotted,purple] (1.5,2.33) -- (1.5,-.3) node[below]{$t_{2}$};
    }
    \onslide<7->{
    \draw[thick,dotted,purple] (4.2,2.33) -- (4.2,-.3) node[below]{$t_{3}$};
    }
    \onslide<8->{
    \draw[thick,dotted,purple] (9.4,1.16) -- (9.4,-.3) node[below]{$t_{4}$};
    }
\end{tikzpicture}
}
%If we assume that $t_{0}=0$,\\
\onslide<5->{$\PHfrappe{a_1}{b_0}{b_1}$ with $r_{1}=\frac{1}{t_{1}-t_{0}}$\\}
\onslide<6->{$\PHfrappe{a_1}{b_1}{b_2}$ with $r_{2}=\frac{1}{t_{2}-t_{1}}$\\}
\onslide<7->{$\PHfrappe{a_0}{b_2}{b_1}$ with $r_{3}=\frac{1}{t_{3}-t_{2}}$\\}
\onslide<8->{$\PHfrappe{a_0}{b_1}{b_0}$ with $r_{4}=\frac{1}{t_{4}-t_{3}}$\\}
\onslide<8->{The formula to estimate the rate of the dynamics of a component according to it TSD is 
\tval{ \Large {$r_{i}=\frac{1}{t_{i}-t_{i-1}}$}}}
\end{column}

\begin{column}{0.3\textwidth}
\scalebox{0.9}{
 \begin{tikzpicture}[scale=0.8]
\path[use as bounding box] (-1,-1) rectangle (2,2);


\TSort{(0,1)}{b}{3}{r}


\only<5->{\TSort{(2,1)}{a}{2}{l}}
\only<5>{
\THit{a_1}{}{b_0}{.east}{b_1}
%\THit{a_0}{out=-120,in=180,selfhit}{a_0}{.west}{a_1}
\path[bounce]
%\TBounce{a_0}{bend left}{a_1}{.south}
\TBounce{b_0}{bend right}{b_1}{.south}
;
\TState{5}{a_1,b_0}
\TState{5}{a_1,b_1}
}

%deuxième estimation
\only<6>{
\THit{a_1}{}{b_1}{.east}{b_2}
%\THit{a_0}{out=-120,in=180,selfhit}{a_0}{.west}{a_1}
\path[bounce]
%\TBounce{a_0}{bend left}{a_1}{.south}
\TBounce{b_1}{bend right}{b_2}{.south}
;
\TState{6}{a_1,b_1}
\TState{6}{a_1,b_2}
}

%troisième  estimation
\only<7>{
\THit{a_0}{}{b_2}{.east}{b_1}
%\THit{a_0}{out=-120,in=180,selfhit}{a_0}{.west}{a_1}
\path[bounce]
%\TBounce{a_0}{bend left}{a_1}{.south}
\TBounce{b_2}{bend right}{b_1}{.north}
;
\TState{7}{a_0,b_2}
\TState{7}{a_0,b_1}
}

%troisième  estimation
\only<8>{
\THit{a_0}{}{b_1}{.east}{b_0}
%\THit{a_0}{out=-120,in=180,selfhit}{a_0}{.west}{a_1}
\path[bounce]
%\TBounce{a_0}{bend left}{a_1}{.south}
\TBounce{b_1}{bend right}{b_0}{.north}
;
\TState{8}{a_0,b_1}
\TState{8}{a_0,b_0}
}

\end{tikzpicture}
}

\end{column}
\end{columns}
\end{frame}







\begin{frame}[c]
  \frametitle{Simulations and analysis}
  
 \begin{center}
  \includegraphics[scale=0.15]{figs/12genes_sim.png}
\end{center}

\textbf{Simulations}

\begin{itemize}
  \item For $r_{a} = r_{i}=10.0$ et $sa = 50 $ for all signalling proteins
  \item With estimated $r$ and $sa$ for MKP3, MKP1, uPAR, Hes5,... according to their expression profiles
  
\end{itemize}

%\textcolor{couleurtheme}{$\Rightarrow$} \fbox{\tval{\large Analyse???}} \textcolor{couleurtheme}{$\Leftarrow$}

\end{frame}




\begin{frame}[c]
  \frametitle{Simulations and analysis}
  
 \begin{center}
  \includegraphics[scale=0.35]{figs/key_nodes1.png}
\end{center}

\textbf{Simulations}

\begin{itemize}
  \item Input node of the system (E\_cadherin)
 \item For biological processes (Cell adhesion, Cell cycle arrest, Keratinocyte differentiation)
  
\end{itemize}

%\textcolor{couleurtheme}{$\Rightarrow$} \fbox{\tval{\large Analyse???}} \textcolor{couleurtheme}{$\Leftarrow$}

\end{frame}


\begin{frame}
   \frametitle{Simulation and Trace analysis}
   %\framesubtitle{work with Guillaume Taupiac}


\begin{columns}
\begin{column}{0.5\textwidth}

\scalebox{0.9}{
\begin{tikzpicture}[scale = 0.8]
       
    \draw[thick, ->] (-.2,-.1)--(7,-.1) node[below]{$t$};
     \foreach \t in {1,2,3,4,5,6}
      \draw[very thick] (\t,-1pt)--(\t,-2pt) node[below,blue]{\small\t};

     \draw[thick, ->] (0,-.2)--(0,3) node[left]{$Level$};
     \foreach \y in {0,1,2}
      \draw[very thick] (2pt,\y)--(-2pt,\y) node[left,blue]{\small\y};
    
    
    \draw[thick,dotted,blue] (0,0) -- (2,0) node[below]{}; 
    \draw[thick,dotted,blue] (2,0) -- (2,1) node[below]{}; 
    \draw[thick,dotted,blue] (2,1) -- (4,1) node[below]{};
    \draw[thick,dotted,blue] (4,1) -- (4,0) node[below]{};
    \draw[thick,dotted,blue] (4,0) -- (6,0) node[below]{};
    \node[instruct,align=center] (mot2) at (4,2) {$\omega=010$};

\end{tikzpicture}
}


%deuxième exemple 
\scalebox{0.9}{
\begin{tikzpicture}[scale = 0.8]
       
    \draw[thick, ->] (-.2,-.1)--(7,-.1) node[below]{$t$};
     \foreach \t in {1,2,3,4,5,6}
      \draw[very thick] (\t,-1pt)--(\t,-2pt) node[below,blue]{\small \t};

     \draw[thick, ->] (0,-.2)--(0,3) node[left]{$Level$};
     \foreach \y in {0,1,2}
      \draw[very thick] (2pt,\y)--(-2pt,\y) node[left,blue]{\small \y};
    
    
    \draw[thick,dotted,blue] (0,0) -- (1,0) node[below]{}; 
    \draw[thick,dotted,blue] (1,0) -- (1,1) node[below]{}; 
    \draw[thick,dotted,blue] (1,1) -- (2,1) node[below]{};
    \draw[thick,dotted,blue] (2,1) -- (2,2) node[below]{};
    \draw[thick,dotted,blue] (2,2) -- (3,2) node[below]{};
    \draw[thick,dotted,blue] (3,2) -- (3,1) node[below]{};
    \draw[thick,dotted,blue] (3,1) -- (5,1) node[below]{};
    \draw[thick,dotted,blue] (5,1) -- (5,0) node[below]{};
    \draw[thick,dotted,blue] (5,0) -- (6,0) node[below]{};
    \node[instruct,align=center] (mot2) at (5,2) {$\omega=01210$};
   

\end{tikzpicture}
}


\end{column}


\begin{column}{0.5\textwidth}

for each component \tval{$C_{i}$, $1 \leq i \leq P$},
\tval{$N$} simulations will generate \tval{$\omega_{i1}, \omega_{i2}, \ldots ,\omega_{iN}$} words.




for $1 \leq j \leq N$

%\newline
%\vspace{1cm}



\scalebox{0.9}{
\begin{tikzpicture}[scale = 0.8]
    
    \node[align=center,blue] (mot) at (2,3) {$\omega_{ij} \Rightarrow $};
    
    \draw[blue] (3,2) rectangle (7,4);

    \node[align=center] (automaton) at (5,3) {$\mathcal{A}_{C_{i}}$};
    
    \node[align=center] (accept) at (8,3) {$\Rightarrow \tval{yes}/\alert{no} $};
    
    \node[align=center] (percent) at (5,1) {\tval{$\% of Acceptance = \frac{\card{YES}}{\card{Simulations}}$}};
   

\end{tikzpicture}
}



\end{column}
\end{columns}
\end{frame}

\begin{frame}
\frametitle{Simulation and Trace analysis}
\begin{tabular}{|c|c||c|c|}
\hline

\textbf{Automate} & \textbf{components} & \textbf{$\%$  validation} & \textbf{$\%$ of acceptance $T_{1}$}
\\ \hline

$\mathcal{A}_{2}(01210)$ & A20 & 91 & 100 
\\ \hline

$\mathcal{A}_{2}(01210)$ & IL1$\_$beta & 81 & 100
\\ \hline

$\mathcal{A}_{2}(01210)$ & IL8 & 93 & 100 
\\ \hline

$\mathcal{A}_{2}(01210)$ & TNF$\_$alpha & 0 & 0
\\ \hline

$\mathcal{A}_{3}(01211)$ & uPar & 76 & 99 
\\ \hline

$\mathcal{A}_{3}(01211)$ & ET1 & 8 & 19 
\\ \hline

$\mathcal{A}_{4}(0121210)$ & DKK1 & 13 & 43

\\ \hline

$\mathcal{A}_{5}(0121211)$ & Hes5 & 0 & 17 
\\ \hline

$\mathcal{A}_{5}(0121211)$ & MKP1 & 9 & 97
\\ \hline

$\mathcal{A}_{6}(0212)$ & SM22 & 11 & 100 
\\ \hline

$\mathcal{A}_{7}(02010)$ & MKP3 & 11 & 98

\\ \hline

$\mathcal{A}_{8}(02121)$ & Tfr & 0 & 94 
\\ \hline

\end{tabular}

\end{frame}




\section{Formal inference of parameters}

\begin{frame}
\frametitle{Formal inference of parameters}


\end{frame}


\section{Discussion}

\begin{frame}
\frametitle{Discussion}

\end{frame}



\begin{frame}[plain,label=title]

% Cadre de titre
\begin{center}
\vspace{1cm}
\setbeamercolor{postit}{fg=black,bg=colortitle}
\begin{beamercolorbox}[sep=0.5em]{postit}
\centering
\Large
\textbf{%
{\normalsize\theconference{}}\\~\\%
\inserttitle
}
\end{beamercolorbox}

% Auteurs et instituts
% Auteurs et instituts


\par
\medskip
\normalsize
Louis Fippo Fitime$^1$
\footnotesize

\texttt{fippofitime@lipn.univ-paris13.fr}

\url{http://www.irccyn.ec-nantes.fr/~fippofit/}



\bigskip
\textbf{Joint work with:} \\  \'Etienne Andr\'e$^1$ and Laure Petrucci$^1$

% Auteurs et instituts

\medskip
\footnotesize
$^1$ LCR / LIPN / CNRS (Paris, France)

\texttt{etienne.andre@lipn.fr}

\texttt{laure.petrucci@lipn.univ-paris13.fr}


\end{center}

Thank you for your attention!!!

\end{frame}



\end{document}


